\documentclass{article}
\usepackage{parskip}
\usepackage{mathtools}
\usepackage{amsthm}

\begin{document}

\title{INF240 Mandatory Assignment 1}
\author{David Huynh}

\maketitle

\section*{Problem 1}

\subsection*{a)}
Fermat's theorem says that for $p$ prime and an integer $a$: $a^{p} \equiv a (mod p)$.\\
If $a$ does not divide $p$ then Fermat's theorem can be stated as: $a^{p-1} \equiv 1 (mod p)$-

To show that $n=55$ is a composite number then we can use the second formula and test different $a$, if $n$ is a composite number then there is some $a$ that the relation does not hold.

\begin{tabular}{l | c | c | r}
$a$ & $a^{p-1} (mod p)$ & $a^{p-1} \equiv 1 (mod p)?$ \\ \hline

1 & 1 & yes\\
2 & $2^{54} (mod 55) = 49$ & no
\end{tabular}

Thus $n=55$ is not prime and therefore composite

\subsection*{b)}

$n=341$ \hspace{5 mm} $k=3$\\
$n-1=340= 2^{2} \cdotp 85$ \hspace{5 mm} $s=2 \hspace{5mm} d=85$

\begin{tabular}{c | c | c | c | c | c}
Iteration & $a$ & $x$ & $x=1$ or $x=n-1?$ & $x_{2}$ & $x_{2}=1$ or $x_{2}=n-1?$\\ \hline

1 & 2 & $2^{85} (mod 341) = 32$ & no & $32^{2} (mod 341) = 1$ & yes\\
2 & 3 & $3^{85} (mod 341) = 254$ & no & $254^{2} (mod 341) = 67$ & no\\
3 & 4 & $4^{85} (mod 341) = 1$ & yes &  &
\end{tabular}

$n=341$ is composite since the algorithm stops for $a=2,4$

\subsection*{c)}

$x^{2} \equiv y^{2} (mod n)$ is a square congruence, from the principle on p.176 in the textbook $gcd(x-y,n)$ is a nontrivial factor of $n$.

Let us use a relation we obtained from b):\\
$32^{2} \equiv 1^{2} (mod$ $341)$\\
gcd(32-1, 341) = 31\\
gcd(32+1, 341) = 11\\
$341 = 11 \cdotp 31$
\end{document}
