\documentclass{article}

\begin{document}

\title{INF252 RA2 review}
\author{David Huynh\\dhu009@student.uib.no}
\date{}

\maketitle

\section{Paper information}
Title: Visualization idioms: A Conceptual Model for Scientific Visualization Systems
Author: Robert B.Haber\\
Author: David A.Mcnabb\\
Year:1990\\
Published as a conference paper, published by IEEE Computer Society Press

\section{Summary}

This RA is about a model for scientific visualization through mappings of data, they define that if there is a specific sequence of mappings it is a visualization idiom.\\
They observe a parallell between numerical simulation and scientific observation/experimentation, these are modeling, solution, interpretation and evaluation. After explaining these phases they explain how this can be extended when dealing with visualization.\\
They write about the hardware and software needed for visualization versus numerical simulations, mentioning the need for powerful specialized hardware and open-source software

\section{Questions}

\subsection*{What did you find interesting about this paper?}

Their section on software and hardware was interesting because they bring up excellent points in pushing open-source and the lack of software being able to sufficently take advantage of the hardware, they mention NCSA's RIVERS project and how they approach software design.

\subsection*{Like/dislike about the paper}

The figures were clear and easy to understand, 
I found the paper a bit too heavy to read because it felt a bit abstract to me.

\section{Context of work}

The paper mentions National Center for Supercomputing Applications, University of Illinois at Urbana-Champaign as contributors to the work described. It was partially funded by National Science Foundation, Apple Computer, Sun Microsystems and Cray Research

\section{Other papers mentioning this paper}

As of 09.09.14 the paper has 430 citations on scholar.google.com.\\
One of these papers is "The State of the Art in Flow Visualization: Feature Extraction and tracking" by Frits H. POst, Benjamin Vrolijk, Helwig Hauser, Robert S. Laramee and Helmut Doleisch. It was published in Computer Graphics Forum, Volume 22, Issue 4, December 2003.
The RA is mentioned in connection to eigenanalysis of the covariance matrix of an object's grid points.

\section{Time used}
For this RA I spent about 2.5h to read the paper and write the RA.
\end{document}