\documentclass{article}

\begin{document}

\title{INF252 Reading Assignment 1}
\author{Huynh, David}
\date{31.08.14}

\maketitle

\section{Paper information}
Title: Perception and Painting: A Search for Effective, Engaging Visualizations\\
Authors: Christopher G. Healey, James T. Enns\\
Year: 2002\\
Published in  "IEEE Computer Graphics and Applications" journal March/April 2002 edition

\section{Summary}
This paper attempts to give answers on how to make scientific visualizations "engaging or aesthetically pleasing" such that the viewer will become more interested in studying the image in detail and making it easier to gain information from it. To do this they take inspiration from the world of art.\\
First they briefly explain how human vision works and they explain how they can use way human vision behaves to craft visualizations that draw the focus to the areas the authors intend.\\
Next they explore nonphotorealistic visualizations and art history, specifically the Impressionist movement, they draw parallells between the way impressionists control color, light and object and the visual features of the authors own visualizations. They then give a strategy for creating painterly visualizations of multidimensional data.\\
Lastly they found that using psychophysical experiments where viewers ranked images based on several different criterias that the majority of their testers prefered realism to abstractionism

\section{Questions}

\subsection*{What did you find interesting about this paper?}
This paper was surprising to me because before reading this paper it did not seem for me that a nonphotorealistic visualization would be able to convey essential information faster and easier than a photorealistic visualization would. 

\subsection*{short question}

Wondering if there are other painting styles that would suit this way of visualization other than impressionist, since I myself is extremely unfamiliar with paintings.

\subsection*{Like/dislike}

I like the large amount of images effectively illustrating the papers findings, makes it easy to understand what they are talking about. Nothing in particular that I disliked

\section{Context of work}

This paper is a summary of findings during the authors past year of labwork and was funded by National Science and Engineering Research Council of Canada and US National Science Foundation Grants IIS-9988507, ACI-0083421

\section{Other papers mentioning this paper}

As of 31.08.14 the paper has 8 citations on scholar.google.com\\
One of these papers is "Human Factors in Visualization Research" by Melanie Tory and Torsten Möller, published in IEE Transactions on Visualization and computer graphics vol 10 No. 1. Jan/Feb 2004\\
This paper writes that the authors in the RA uses the phenomenom in the RA (they call it pre-attentative visualization) to improve glyph-based multivariate data displays.

\section{Time used}
I used about 1.5h to read the RA and answer the questions

\end{document}