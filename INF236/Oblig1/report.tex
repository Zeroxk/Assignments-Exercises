\documentclass{article}
\usepackage{parskip}
\usepackage{mathtools}

\begin{document}

\title{INF236\\ Compulsary Assignment 1}
\author{David Huynh\\ E-mail: dhu009@student.uib.no}
\date{}

\maketitle

\section*{Problem 1}

\subsection*{Solution}

I divide the workload by splitting n into $n/np$ intervals and each process finds the consecutive odd primes for its interval. At the end I use \verb|MPI_Reduce| to compute the total number of pairs.

\subsection*{Benchmarks}

\begin{table}[h]
\begin{tabular}{llll}
$np/n$ & $10^{6}$ & $10^{7}$ & $10^{8}$\\
5 & 0.2880 & 7.1110 & 175.92\\
10 & 0.1274 & 1.918 & 41.01\\
20 & 0.0617 & 1.017 & 20.8112\\
40 & 0.05297 & 0.7910 & 10.858
\end{tabular}
\end{table}

\section*{Problem 2}

\subsection*{Solution}

I use row granularity for the dynamic task assignment, to each idle slave I send a row index $i$ and keep track of which row process $p$ is working on. When $p$ is finished I insert the results directly into the 2D results table and give it a new job if there are any availble.

\subsection*{Benchmarks}

\begin{table}[h]
\begin{tabular}{llll}
 & Static & & \\
$np/nxm$ & 200x$10^{6}$ & 300x$10^{6}$ &\\
1 & 47.395 & 107.734 &\\
10 & 17.323 & 39.826 &\\
20 & 12.834 & 33.952 &
\end{tabular}
\end{table}

Dynamic was faster than static.

\end{document}
