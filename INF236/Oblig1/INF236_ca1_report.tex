\documentclass{article}
\usepackage{parskip}
\usepackage{mathtools}

\begin{document}

\title{INF236\\ Compulsary Assignment 1}
\author{David Huynh\\ E-mail: dhu009@student.uib.no}
\date{}

\maketitle

\section*{Problem 1: Consecutive prime pairs}

\subsection*{Solution}

I divide the workload by splitting n into $n/np$ intervals and each process finds the consecutive odd primes for its interval. At the end I use \verb|MPI_Reduce| to compute the total number of pairs.

\subsection*{Benchmarks}

All times in seconds
\begin{table}[h]
\begin{tabular}{llll}
$np/n$ & $1\cdotp10^{6}$ & $1\cdotp10^{7}$ & $1\cdotp10^{8}$\\
5 & 0.2880 & 7.1110 & 175.92\\
10 & 0.1274 & 1.918 & 41.01\\
20 & 0.0617 & 1.017 & 20.8112\\
40 & 0.05297 & 0.7910 & 10.858\\
60 & 0.0392 & 0.7827 & 10.0781\\
80 & 0.0344  & 0.6715 & 9.4526
\end{tabular}
\end{table}

\section*{Problem 2: Static vs dynamic task assignment of finding the Mandelbrot set}

\subsection*{Solution}

I use row granularity for the dynamic task assignment, to each idle slave I send a row index $i$ and keep track of which row process $p$ is working on. When $p$ is finished I insert the results directly into the 2D results table and give it a new row index to work on and update which row $p$ is working on. The slaves terminate when there are no more row indices to work on, which means we have completed the Mandelbrot set.

\subsection*{Benchmarks}

For static benchmarking I just used the supplied "mandel.c" shown in lecture.

All times in seconds
\begin{table}[h]
\begin{tabular}{lllll}
Static task assignment: & & & & \\
$np/mxn$ & 200x$10^{6}$ & 300x$10^{6}$ & 200x$10^{7}$ & 300x$10^{7}$\\
2 & 22.2150 & 49.8010 & 216.6573 & 501.7430\\
5 & 30.8100 & 69.1648 & 306.4707 & 687.4342\\
10 & 15.5455 & 34.9789 & 153.3336  & 344.4189\\
20 & 9.2147 & 20.4410 & 88.1760 & 200.3842\\
40 & 5.2880 & 12.6460 & 29.2556  & 79.8056\\
80 & 3.2068 & 6.6983 & 28.8200 & 63.4330
\end{tabular}
\end{table}

\begin{table}[h]
\begin{tabular}{lllll}
Dynamic task assignment: & & & &\\
$np/mxn$ & 200x$10^{6}$ & 300x$10^{6}$ & 200x$10^{7}$ & 300x$10^{7}$\\
2 & 42.8625 & 96.4700 & 425.9642& 962.7363\\
5 & 11.0000 & 24.5547 & 107.2900 & 242.3183\\
10 & 5.1150 & 11.0881 & 47.5978 & 107.2945\\
20 & 2.5944 & 5.6562 & 23.4203 & 53.6336\\
40 & 1.6073 & 3.0783 & 11.9102 & 26.3538\\
80 & 0.3128 & 2.2228 & 1.8048 & 16.8360
\end{tabular}
\end{table}

Dynamic being slower than static when $np=2$ is probably because there will only be 1 slave in my program at that time so would be the same as $np=1$ but with communication overhead. In all other cases dynamic task assignment wins by atleast a factor of 2.

\end{document}
