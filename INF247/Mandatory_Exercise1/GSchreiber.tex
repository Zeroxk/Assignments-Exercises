
\documentclass{article}
\usepackage{parskip}
\usepackage{mathtools}

\begin{document}

\title{INF247\\ Mandatory Exercise 1\\ G-Schreiber Cryptanalysis}
\author{David Huynh\\ E-mail: dhu009@student.uib.no}
\date{}

\maketitle

\section{Description, known data}

We are given:
\begin{itemize}
	\item 1274 ciphertext chars
	\item Ciphertext chars represented by their position in the CCITT2 code
	\item 1300 plaintext chars
	\item Plaintext consists of latin alphabet applied 50 times
\end{itemize}

Given these pieces of data, find:
\begin{itemize}
	\item All internal settings of the cipher
	\item Initial state of the wheels
	\item Cabling permutation
	\item The remaining 26 ciphertext chars
\end{itemize}

\section{Solution}

With the given data we can perform a known plain-text attack, we can use these properties:
\begin{enumerate}
	\item If the ciphertext group is 11111 or 00000 then key-stream $\oplus$  plain-text = 11111 or 00000 (switch input) because no matter what the controlbit group is the ciphertext will be the same.\\ 
This means we can $\oplus$ ciphertext with related plain-text to aquire the 5-bit key-stream sequence.
	\item If switch input and output are of weight 1 or 4 then all valid controlbit groups for that input/output will have common bits at certain positions
\end{enumerate}

By using each of these properties we can find the cabling permutation and thus initial state of the cipher

\subsection{Key-stream}

To find key-stream we
\begin{enumerate}
	\item Find all occurences where ciphertext char is 11111 or 00000
	\item Note at which position in the ciphertext it occurs
	\item $\oplus$ ciphertext group with plain-text group
	\item Store position of ciphertext together with the xor'd group
\end{enumerate}

Next we need to find the position of 5 wheels at time $t$, we do this by:
\begin{enumerate}
	\item Do a type of bucket sort on each wheel, with wheelSize number of buckets and each bucket represents time $t$, or position of the wheel.
	\item In each bucket we store what the key-stream was at that moment
\end{enumerate}

After all the wheels have been sorted we compare each wheel at same time $t$ and look at the first column of the bucket. We look for contradictions in the column, if the columns do not have all 0s or all 1s then wheel $w$ could not have that position at $t$, further study lets us know that there is only one position that $w$ could be in, because all the other possibilities give contradictions.

\subsection{Controlbit groups}

We find the valid control bit groups of the switch input/outputs of weight 1 and 4 by:
\begin{enumerate}
	\item Generate all possible control bit groups, $2^{5} = 32$ groups
	\item Run all controlbit groups on the inputs and check if it gives desired output
	\item If yes, store it
	\item Check the valid controlbit groups, see if there are any patterns
	\item Use this pattern to run the same comparison algorithm on wheels we used in section 2.1
\end{enumerate}

Though patterns do not give complete information, by process of elmination and knowledge of where the other bits are cabled to we can find the rest of the cabling permutation

\end{document}