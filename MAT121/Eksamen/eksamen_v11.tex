\documentclass{article}

\usepackage{mathtools}
\usepackage{parskip}
\usepackage{amsfonts}

\begin{document}

\title{Eksamen MAT121 Våren 2011}

\maketitle

\section*{Oppgave 1}

\subsection*{1}

$A=[
\begin{matrix}
    1 & 2 & 1 & 4\\
    2 & 2 & 4 & 7\\
    -1 & 0 & -3 & -3
\end{matrix}
, \hspace{5 mm}
b=
\begin{matrix}
    6\\
    14\\
    b
\end{matrix}
$

\subsubsection*{a}
Redusert trappeform av $A$:

-2*rad1 + rad2\\
$
\begin{matrix}
    1 & 2 & 1 & 4\\
    0 & -2 & 2 & -1\\
    -1 & 0 & -3 & -3
\end{matrix}
$

rad1+rad3\\
$
\begin{matrix}
    1 & 2 & 1 & 4\\
    0 & -2 & 2 & -1\\
    0 & 2 & -2 & 1
\end{matrix}
$

rad2/-2\\
$
\begin{matrix}
    1 & 2 & 1 & 4\\
    0 & 1 & -1 & \frac{1}{2}\\
    0 & 2 & -2 & 1
\end{matrix}
$

-2*rad2 + rad1\\
$
\begin{matrix}
    1 & 0 & 3 & 3\\
    0 & 1 & -1 & \frac{1}{2}\\
    0 & 2 & -2 & 1
\end{matrix}
$

-2*rad2 + rad3\\
$
\begin{matrix}
    1 & 0 & 3 & 3\\
    0 & 1 & -1 & \frac{1}{2}\\
    0 & 0 & 0 & 0
\end{matrix}
$

\subsubsection*{b}

Når $b !=-8$ finnes det ingen løsninger pga vi får en nullrad\\
Når $b = -8$ finnes det uendelig mange løsninger pga en frivariabel\\
Finnes aldri kun en løsning fordi det finnes alltid frivariabel

Generell løsning ved $b=-8$:\\
$x= [
\begin{matrix}
    x_{1}\\
    x_{2}\\
    x_{3}\\
    x_{4}
\end{matrix}
]= [
\begin{matrix}
    8\\
    -1\\
    0\\
    0
\end{matrix}
] +x_{3}[
\begin{matrix}
    -3\\
    1\\
    1\\
    0
\end{matrix}
] +x_{4}[
\begin{matrix}
    -3\\
    -\frac{1}{2}\\
    0\\
    1
\end{matrix}
]
$

\subsubsection*{c}

Null(A): løser $Ax=0$\\
$x_{1} = -3x_{3}-3x_{4}\\
x_{2} = x_{3} - \frac{1}{2}x_{4}\\
x_{3} = 0\\
x_{4} = 0
$

$Null(A): span\{
\left [
\begin{matrix}
    -3\\
    1\\
    1\\
    0
\end{matrix}
\right ], \hspace{5 mm}
\left [
\begin{matrix}
    -3\\
    -\frac{1}{2}\\
    0\\
    1
\end{matrix}
\right ]
\}
$
Dimension Null(A): 2

Col(A): Pivot kolonner fra rad redusert $A$, kolonne 1 og 2\\
$Col(A) = \{ 
\left [
\begin{matrix}
    1\\
    2\\
    -1
\end{matrix}
\right ], \hspace{5 mm}
\left [
\begin{matrix}
    2\\
    2\\
    0
\end{matrix}
\right ] \}
$
Dimensjon til Col(A): 2

$Rank(A) = n-dim(Null(A)) = 4-2 = 2$

Vi finner ortonormal basis med Gram-Schmidt prosessen.

Ortonormal basis for Col(A):\\
$x_{1} = \{1,2-1\}^{T}$\\
$x_{2} = \{2,2,0\}^{T}$

$v_{1} = x_{1}$\\
$v_{2} = x_{2} - \frac{x_{2}v_{1}}{v_{1}v_{1}}v_{1}$\\
$v_{2} = \{2,2,0\} - \frac{ \{2,2,0\}\{1,2,-1\}}{\{1,2,-1\}^{2}}\{1,2,-1\}$\\
$v_{2} = \{2,2,0\} - \frac{ 2+4+0 }{ 1 + 4 + 1 }\{1,2,-1\}$\\
$v_{2} = \{2,2,0\} - \frac{ 6 }{ 6 }\{1,2,-1\}$\\
$v_{2} = \{2,2,0\} - \{1,2,-1\}$\\
$v_{2} = \{1,0,1\}$\\

Normaliser og vi får ortonormal basis:
Col(A) = Span$\{ \frac{1}{\sqrt{6}}
\left [
\begin{matrix}
    1\\
    2\\
    -1
\end{matrix}
\right ]
, \frac{1}{ \sqrt{2} }
\left [
\begin{matrix}
    1\\
    0\\
    1
\end{matrix}
\right ]
\}
$

Ortonormal basis for Null(A):\\
$x_{1} = \{-3,1,1,0\}$\\
$x_{2} = \{-3,-\frac{1}{2},0,1\}$

$v_{1} = x_{1}$\\
$v_{2} = x_{2} - \frac{x_{2}v_{1}}{v_{1}v_{1}}v_{1}$\\
$v_{2} = \{-3,-\frac{1}{2},0,1\} - \frac{ \{-3,-\frac{1}{2},0,1\}\{-3,1,1,0\}}{ \{-3,1,1,0\}^{2}}\{-3,1,1,0\}$\\
$v_{2} =  \{-3,-\frac{1}{2},0,1\} - \frac{9-\frac{1}{2}}{9+1+1}\{-3,1,1,0\}$\\
$v_{2} = \{-3,-\frac{1}{2},0,1\} - \{\frac{17}{22}\}\{-3,1,1,0\}$\\
$v_{2} = \{-3,-\frac{1}{2},0,1\} - \{\frac{51}{22}, \frac{17}{22}, \frac{17}{22}, 0\}$\\
$v_{2} = \{-\frac{15}{22}, -\frac{28}{22}, -\frac{17}{22}, 1\}$\\
$v_{2} = \{-\frac{15}{22}, -\frac{14}{11}, -\frac{17}{22}, 1\}$\\

Normaliser og vi får ortonormal basis for Null(A) = Span$\{\frac{1}{\sqrt{11}}
\left [
\begin{matrix}
    -3\\
    1\\
    1\\
    0
\end{matrix}
\right ],
\frac{1}{||v||}
\left [
\begin{matrix}
    -\frac{15}{22}\\
    -\frac{14}{11}\\
    -\frac{17}{22}\\
    1
\end{matrix}
\right ]
$
\}

\subsection*{2}

$B=
\left [
    \begin{matrix}
        1 & 1 & 0\\
        1 & 1 & 0\\
        0 & 0 & -1
    \end{matrix}
\right ]
, \hspace{5 mm}
C=
\left [
    \begin{matrix}
        1 & 1 & 0\\
        0 & 1 & 0\\
        0 & 0 & 1
    \end{matrix}
\right ]
$

$B$ har to like kolonner, derfor er det ikke et lineært uavhengig subset og $B$ er ikke inverterbar.

$C$ er inverterbar fordi $detC=1$ (kan ses fra C er triangelmatrise), regner inverse med gauss eliminasjon.
$[C|I]=
\left [
    \begin{matrix}
        1 & 1 & 0\\
        0 & 1 & 0\\
        0 & 0 & 1
    \end{matrix}
\left |
    \begin{matrix}
        1 & 0 & 0\\
        0 & 1 & 0\\
        0 & 0 & 1
    \end{matrix}
\right .
\right ]
$
\textasciitilde
$
\left [
    \begin{matrix}
        1 & 0 & 0\\
        0 & 1 & 0\\
        0 & 0 & 1
    \end{matrix}
\left |
    \begin{matrix}
        1 & -1 & 0\\
        0 & 1 & 0\\
        0 & 0 & 1
    \end{matrix}
\right .
\right ]
$

\subsection*{3}

Egenverdier og egenvektorer til B:\\
$
\left [
    \begin{matrix}
        1-\lambda & 1 & 0\\
        1 & 1-\lambda & 0\\
        0 & 0 & -1-\lambda
    \end{matrix}
\right ]
$
\end{document}
