\documentclass{article}

\usepackage{parskip}
\usepackage{mathtools}
\usepackage{amsfonts}
\usepackage{enumerate}

\begin{document}

\section*{Exercise 1}

\subsection*{a}
$A=
\left (
    \begin{matrix}
        1 & 2 & 0 & 2\\
        0 & 1 & 1 & 1\\
        0 & -2 & 1 & 1\\
        1 & 2 & 1 & 3
    \end{matrix}
\right )
$
 $\vec{b}=
\left (
    \begin{matrix}
        5\\
        3\\
        0\\
        7
    \end{matrix}
\right )
$

$B=
\left [
        \begin{matrix}
            1 & 2 & 0 & 2 & 5\\
            0 & 1 & 1 & 1 & 3\\
            0 & -2 & 1 & 1 & 0\\
            1 & 2 & 1 & 3 & 7
        \end{matrix}
\right ]
$

Add $-1*row1$ to $row4$\\
$
\left [
        \begin{matrix}
            1 & 2 & 0 & 2 & 5\\
            0 & 1 & 1 & 1 & 3\\
            0 & -2 & 1 & 1 & 0\\
            0 & 0 & 1 & 1 & 2
        \end{matrix}
\right ]
$

Add $-2*row2$ to $row1$\\
$
\left [
        \begin{matrix}
            1 & 0 & -2 & 0 & -1\\
            0 & 1 & 1 & 1 & 3\\
            0 & -2 & 1 & 1 & 0\\
            0 & 0 & 1 & 1 & 2
        \end{matrix}
\right ]
$

Add $2*row2$ to $row3$\\
$
\left [
        \begin{matrix}
            1 & 0 & -2 & 0 & -1\\
            0 & 1 & 1 & 1 & 3\\
            0 & 0 & 3 & 3 & 6\\
            0 & 0 & 1 & 1 & 2
        \end{matrix}
\right ]
$

Divide $row3$ by 3\\
$
\left [
        \begin{matrix}
            1 & 0 & -2 & 0 & -1\\
            0 & 1 & 1 & 1 & 3\\
            0 & 0 & 1 & 1 & 2\\
            0 & 0 & 1 & 1 & 2
        \end{matrix}
\right ]
$

Add $2*row3$ to $row1$\\
$
\left [
        \begin{matrix}
            1 & 0 & 0 & 2 & 3\\
            0 & 1 & 1 & 1 & 3\\
            0 & 0 & 1 & 1 & 2\\
            0 & 0 & 1 & 1 & 2
        \end{matrix}
\right ]
$

Add $-1*row3$ to $row2$\\
$
\left [
        \begin{matrix}
            1 & 0 & 0 & 2 & 3\\
            0 & 1 & 0 & 0 & 1\\
            0 & 0 & 1 & 1 & 2\\
            0 & 0 & 1 & 1 & 2
        \end{matrix}
\right ]
$

Add $-1*row3$ to $row4$\\
$
\left [
        \begin{matrix}
            1 & 0 & 0 & 2 & 3\\
            0 & 1 & 0 & 0 & 1\\
            0 & 0 & 1 & 1 & 2\\
            0 & 0 & 0 & 0 & 0
        \end{matrix}
\right ]
$

Pivot columns are 1,2 and 3.\\
Pivot positions are $B_{1,1}$, $B_{2,2}$, $B_{3,3}$. 

Solving $A\Vec{x}=\Vec{b}$. We look at the reduced echelon form of $B$ and that gives us this system:
\begin{flalign*}
x_{1} & & + 2x_{4} &= 3 &\\
    &x_{2} & &= 1\\
      & &x_{3} + x_{4} &= 2 &\\
\end{flalign*}

Gives us solution:
\begin{flalign*}
x_{1} &= 3-2x_{4} &\\
x_{2} &= 1 &\\
x_{3} &= 2-x_{4} &\\
x_{4} &= free
\end{flalign*}

\subsection*{b}

\subsubsection*{Basis and dimension for Col(B)}
Basis: $\{ \{1,0,0,0\}, \{0,1,0,0\}, \{0,0,1,0\} \}$\\
Dimension: by definition of dimension(p. 193 in our book) dim(Col(B)) =\# of vectors in basis = 3

\subsubsection*{Basis and dimension for Row(B)}
Basis: $\{ \{1,0,0,2,3\}, \{0,1,0,0,1\}, \{0,0,1,1,2\}\}\\$
Dimension: by definition of dimension(p. 193 in our book) dim(Row(B)) = dim(Col(B)) = 3

\subsubsection*{Basis and dimension for Null(B)}
Basis: $\{ \{-2, 0, -1, 1\}, \{-3,-1,-2,0,1\} \}$\\
Solve $Bx=0$, we already have reduced echelon form of $A$ by $B$ so just add column in $B$ with all 0s\\

$
\left [
        \begin{matrix}
            1 & 0 & 0 & 2 & 3 & 0\\
            0 & 1 & 0 & 0 & 1 & 0\\
            0 & 0 & 1 & 1 & 2 & 0\\
            0 & 0 & 0 & 0 & 0 & 0
        \end{matrix}
\right ]
$

$
\left [
    \begin{matrix}
        -2x_{4}-3x_{5}\\
         -x_{5}\\
         -x_{4}-2x_{5}\\
         x_{4}\\
         x_{5}
    \end{matrix}
\right ]
$

Decomposition:
$
\left [
    \begin{matrix}
        -2x_{4}-3x_{5}\\
         -x_{5}\\
         -x_{4}-2x_{5}\\
         x_{4}\\
         x_{5}
    \end{matrix}
\right ]
$
$ = x_{4} 
\left [
    \begin{matrix}
        -2\\
        0\\
         -1\\
         1\\
         0
    \end{matrix}
\right ]
+ x_{5}
\left [
    \begin{matrix}
        -3\\
        -1\\
        -2\\
        0\\
        1
    \end{matrix}
\right ]
$

Dimension: 2

\subsubsection*{Rank of B}
By The Rank Theorem (p.194 in our book)
\begin{flalign*}
    rank(B) + dim(Null(B)) &= n &\\
    rank(B) &= n-dim(Null(B))\\
    rank(B) &= 5-2\\
    rank(B) &= 3
\end{flalign*}

\subsection*{c}

A matrix $A$ is invertible if its row-reduced echelon form is $I_{4}$, we already know the row-reduced echelon form  of $B$ so row-reduced echelon form of $A$ is just $B$ without the last column. This is because while row-reducing $B$ you also row-reduce $A$.

RREF of $A$:\\
$A=
\left [
    \begin{matrix}
        1 & 0 & 0 & 2\\
        0 & 1 & 0 & 0\\
        0 & 0 & 1 & 1\\
        0 & 0 & 0 & 0
    \end{matrix}
\right ]
$

We see that $A \not = I_{4}$ so $A$ is not invertible.

\newpage

\section*{Exercise 2}

\subsection*{a}

$A=
\left (
    \begin{matrix}
        1 & -1 & 2\\
        0 & 1 & 3
    \end{matrix}
\right )
,
B=
\left (
    \begin{matrix}
        2 & -1\\
        -1 & 0\\
        3 & -1
    \end{matrix}
\right )
, \vec{x} =
\left (
    \begin{matrix}
        1\\
        2
    \end{matrix}
\right )
$

\subsubsection*{y=B$\vec{x}$}
$B\vec{x} =
\left (
    \begin{matrix}
        2 & -1\\
        -1 & 0\\
        3 & -1
    \end{matrix}
\right )
\left (
    \begin{matrix}
        1\\
        2
    \end{matrix}
\right )
=
\left (
    \begin{matrix}
        2 + (-2)\\
        -1 + 0\\
        3 - 2
    \end{matrix}
\right )
=
\left (
    \begin{matrix}
        0\\
        -1\\
        1
    \end{matrix}
\right )
$\\
$\vec{y} =
\left (
    \begin{matrix}
        0\\
        -1\\
        1
    \end{matrix}
\right )
$

\subsubsection*{C=AB}
$AB=
\left (
    \begin{matrix}
        1 & -1 & 2\\
        0 & 1 & 3
    \end{matrix}
\right )
\left (
    \begin{matrix}
        2 & -1\\
        -1 & 0\\
        3 & -1
    \end{matrix}
\right )
=
\left (
    \begin{matrix}
        1*2 + (-1)*(-1) + 2*3 & 1*(-1) + (-1)*0 + 2*(-1)\\
        0*2 + 1*(-1) + 3*3 & 0*(-1) + 1*0 + 3*(-1)
    \end{matrix}
\right )
=
\left (
    \begin{matrix}
        2 + 1 + 6 & -1 + 0 - 2\\
        0 - 1 + 9 & 0 + 0 - 3
    \end{matrix}
\right )
=
\left (
    \begin{matrix}
        9 & -3\\
        8 & -3
    \end{matrix}
\right )
$

$C=
\left (
    \begin{matrix}
        9 & -3\\
        8 & -3
    \end{matrix}
\right )
$

\subsubsection*{A$\vec{y}$}
$A\vec{y} =
\left (
    \begin{matrix}
        1 & -1 & 2\\
        0 & 1 & 3
    \end{matrix}
\right )
\left (
    \begin{matrix}
        0\\
        -1\\
        1
    \end{matrix}
\right )
=
\left (
    \begin{matrix}
        0 + (-1)*(-1) + 2*1\\
        0 + (-1)*1 + 3*1
    \end{matrix}
\right )
=
\left (
    \begin{matrix}
        0 + 1 + 2\\
        0 - 1 + 3
    \end{matrix}
\right )
=
\left (
    \begin{matrix}
        3\\
        2
    \end{matrix}
\right )
$

$A\vec{y}=
\left (
    \begin{matrix}
        3\\
        2
    \end{matrix}
\right )
$

\subsubsection*{C$\vec{x}$}
$C\vec{x} =
\left (
    \begin{matrix}
        9 & -3\\
        8 & -3
    \end{matrix}
\right )
\left (
    \begin{matrix}
        1\\
        2
    \end{matrix}
\right )
=
\left (
    \begin{matrix}
        9 + (-3)*2\\
        8 + (-3)*2
    \end{matrix}
\right )
=
\left (
    \begin{matrix}
        9 - 6\\
        8 - 6
    \end{matrix}
\right )
=
\left (
    \begin{matrix}
        3\\
        2
    \end{matrix}
\right )
$

$C\vec{x}=
\left (
    \begin{matrix}
        3\\
        2
    \end{matrix}
\right )
$

\subsubsection*{C$^{T}$B$^{T}$}
C$^{T}=
\left (
    \begin{matrix}
        9 & 8\\
        -3 & -3
    \end{matrix}
\right )
$,
B$^{T}=
\left (
    \begin{matrix}
        2 & -1 & 3\\
        -1 & 0 & -1\\
    \end{matrix}
\right )
$

C$^{T}$B$^{T}=
\left (
    \begin{matrix}
       9*2 + 8*(-1) & 9*(-1) + 0 & 9*3 + 8*(-1)\\
       -3*2 + (-3)*(-1) & (-3)*(-1) + 0 & (-3)*3 + (-3)*(-1)
    \end{matrix}
\right )
=
\left (
    \begin{matrix}
        18-8 & -9 & 27-8\\
        -6+3 & 3 & -9+3
    \end{matrix}
\right )
=
\left (
    \begin{matrix}
        10 & -9 & 19\\
        -3 & 3 & -6
    \end{matrix}
\right )
$

C$^{T}$B$^{T}=
\left (
    \begin{matrix}
        10 & -9 & 19\\
        -3 & 3 & -6
    \end{matrix}
\right )
$

\subsubsection*{(A$^{T}$+B)C}
A$^{T}=
\left (
    \begin{matrix}
        1 & 0\\
        -1 & 1\\
        2 & 3
    \end{matrix}
\right )
$

A$^{T}C=
\left (
    \begin{matrix}
        1 & 0\\
        -1 & 1\\
        2 & 3
    \end{matrix}
\right )
\left (
    \begin{matrix}
        9 & -3\\
        8 & -3
    \end{matrix}
\right )
=
\left (
    \begin{matrix}
        9 + 0 & -3 + 0\\
        -9 + 8 & 3 - 3\\
        18 + 24 & -6 - 9
    \end{matrix}
\right )
=
\left (
    \begin{matrix}
        9 & -3\\
        -1 & 0\\
        42 & -15
    \end{matrix}
\right )
$

We notice that $(C^{T}B^{T})^{T} = BC$ by transpose properties $(A^{T})^{T}=A$ and $(AB)^{T}=B^{T}A^{T}$.\\
$(C^{T}B^{T})^{T} = BC =
\left (
    \begin{matrix}
        10 & -9 & 19\\
        -3 & 3 & -6
    \end{matrix}
\right )
^{T}=
\left (
    \begin{matrix}
        10 & -3\\
        -9 & 3\\
        19 & -6
    \end{matrix}
\right )
$

A$^{T}$C + BC =
$
\left (
    \begin{matrix}
        9 & -3\\
        -1 & 0\\
        42 & -15
    \end{matrix}
\right )
+
\left (
    \begin{matrix}
        10 & -3\\
        -9 & 3\\
        19 & -6
    \end{matrix}
\right )       
=
\left (
    \begin{matrix}
        19 & -6\\
        -10 & 3\\
        61 & -21
    \end{matrix}
\right )
$

\subsection*{b}

\subsubsection*{$A^{2}$?}
We cannot calculate $A^{2}$ because to multiply two matrices $A$ with dimensions $m \times k$ the second matrix $B$ needs to have dimensions $k \times n$ by theorem. In this exercise $B=A$ and since $A$ is not a square matrix it will not meet the requirements.

\subsubsection*{$A^{-1}$?}
We cannot calculate $A^{-1}$ because $A$ is not square, therefore $detA$ is not defined and by definition of determinant (p.203 in our book) it does not have an inverse

\subsection*{c}
We denote the matrix as $A$.\\
$A=
\left (
    \begin{matrix}
        1 & 1 & 2\\
        -1 & 1 & 1\\
        2 & 3 & 3
    \end{matrix}
\right )
$

We apply formula for inverse of $n \times n$ matrix:\\
$A^{-1} = \frac{1}{det(A)} adj(A)$

To find $adj(A)$ we need to find transpose of the cofactor-matrix:

Cofactor-matrix $C$:\\
$A_{11}=
\left |
    \begin{matrix}
        1 & 1\\
        3 & 3
    \end{matrix}
\right |
, A_{12} =
\left |
    \begin{matrix}
        -1 & 1\\
        2 & 3
    \end{matrix}
\right |
, A_{13} =
\left |
    \begin{matrix}
        -1 & 1\\
        2 & 3
    \end{matrix}
\right |
$

$A_{21}=
\left |
    \begin{matrix}
        1 & 2\\
        3 & 3
    \end{matrix}
\right |
, A_{22}=
\left |
    \begin{matrix}
        1 & 2\\
        2 & 3
    \end{matrix}
\right |
, A_{23}=
\left |
    \begin{matrix}
        1 & 1\\
        2 & 3
    \end{matrix}
\right |
$

$A_{31}=
\left |
    \begin{matrix}
        1 & 2\\
        1 & 1
    \end{matrix}
\right |
, A_{32}=
\left |
    \begin{matrix}
        1 & 2\\
        -1 & 1
    \end{matrix}
\right |
, A_{33}=
\left |
    \begin{matrix}
        1 & 1\\
        -1 & 1
    \end{matrix}
\right |
$

$
\left (
    \begin{matrix}
        + & - & +\\
        - & + & -\\
        + & - & +
    \end{matrix}
\right )
$

$C=
\left (
    \begin{matrix}
        \left |
            \begin{matrix}
                1 & 1\\
                3 & 3
            \end{matrix}
        \right |
        = + 0,
        \left |
            \begin{matrix}
                -1 & 1\\
                2 & 3
            \end{matrix}
        \right |
        = - (-5),
        \left |
            \begin{matrix}
                -1 & 1\\
                2 & 3
            \end{matrix}
        \right |
        = + (-5)\\
        \left |
            \begin{matrix}
                1 & 2\\
                3 & 3
            \end{matrix}
        \right |
        = - (-3),
        \left |
            \begin{matrix}
                1 & 2\\
                2 & 3
            \end{matrix}
        \right |
        = + (-1),
        \left |
            \begin{matrix}
                1 & 1\\
                2 & 3
            \end{matrix}
        \right |
        = - 1\\
        \left |
            \begin{matrix}
                1 & 2\\
                1 & 1
            \end{matrix}
        \right |
        = + (-1),
        \left |
            \begin{matrix}
                1 & 2\\
                -1 & 1
            \end{matrix}
        \right |
        = - 3,
        \left |
            \begin{matrix}
                1 & 1\\
                -1 & 1
            \end{matrix}
        \right |
        = + 2
    \end{matrix}
\right )
=\\
\left (
    \begin{matrix}
        0 & 5 & -5\\
        3 & -1 & -1\\
        -1 & -3 & 2
    \end{matrix}
\right )
$

We calculate det$A$ next by multiplying each element at index $i$ in row $A_{1}$ with the element in index $i$ in row $C_{1}$ and summing them together:\\
det$A = (1*0) + (1*5) + (2*-5) = 0 + 5 - 10 = -5$

adj(A) = C$^{T}$
C$^{T}=
\left (
    \begin{matrix}
        0 & 3 & -1\\
        5 & -1 & -3\\
        -5 & -1 & 2
    \end{matrix}
\right )
$

A$^{-1} = \frac{1}{-5}
\left (
    \begin{matrix}
        0 & 3 & -1\\
        5 & -1 & -3\\
        -5 & -1 & 2
    \end{matrix}
\right )       
=
\left (
    \begin{matrix}
        0 & \frac{3}{-5} & \frac{1}{5}\\
        -1 & \frac{1}{5} & \frac{3}{5}\\
        1 & \frac{1}{5} & \frac{2}{-5}
    \end{matrix}
\right )
$

A$^{-1} =
\left (
    \begin{matrix}
        0 & \frac{3}{-5} & \frac{1}{5}\\
        -1 & \frac{1}{5} & \frac{3}{5}\\
        1 & \frac{1}{5} & \frac{2}{-5}
    \end{matrix}
\right )
$
\newpage

\section*{Exercise 3}

\subsection*{a}

To find the volume of the parallelepiped we need to find the scalar triple product (theorem 9, p. 221 in our book). We can do this by computing the absolute value of the determinant

$
\left |
    \begin{matrix}
        -1 & 0 & 2\\
        3 & -1 & 3\\
        4 & 0 & -1
    \end{matrix}
\right |
=-1
\left |
    \begin{matrix}
        -1 & 3\\
        0 & -1
    \end{matrix}
\right |
-0
\left |
    \begin{matrix}
        3 & 3\\
        4 & -1
    \end{matrix}
\right |
+2
\left |
    \begin{matrix}
        3 & -1\\
        4 & 0
    \end{matrix}
\right |
=
-1 + 0 + (2*4)
=
|7| = 7
$

The volume of the parallelpiped is 7.

\subsection*{b}

Let us view the system as Ax=b

$A=
\left (
    \begin{matrix}
        2 & 1 & 1\\
        -1 & 0 & 2\\
        3 & 1 & 3
    \end{matrix}
\right )
, \hspace{5 mm}
b=
\left (
    \begin{matrix}
        4\\
        2\\
        -2
    \end{matrix}
\right )
$

$A_{1}(b)=
\left (
    \begin{matrix}
        4 & 1 & 1\\
        2 & 0 & 2\\
        -2 & 1 & 3
    \end{matrix}
\right )
,\hspace{5 mm}
A_{2}(b) =
\left (
    \begin{matrix}
        2 & 4 & 1\\
        -1 & 2 & 2\\
        3 & -2 & 3
    \end{matrix}
\right )
,\hspace{5 mm} 
A_{3}(b) =
\left (
    \begin{matrix}
        2 & 1 & 4\\
        -1 & 0 & 2\\
        3 & 1 & -2
    \end{matrix}
\right )
$

Next we need to find determinants.\\
$|detA| = 2
\left |
    \begin{matrix}
        0 & 2\\
        1 & 3
    \end{matrix}
\right |
- 1
\left |
    \begin{matrix}
        -1 & 2\\
        3 & 3
    \end{matrix}
\right |
+ 1
\left |
    \begin{matrix}
        -1 & 0\\
        3 & 1
    \end{matrix}
\right |
=
2*(-2) - (-9) + (-1)
=
-4 + 9 - 1
=
4
$

$|detA_{1}(b)|= 4
\left |
    \begin{matrix}
        0 & 2\\
        1 & 3
    \end{matrix}
\right |
- 1
\left |
    \begin{matrix}
        2 & 2\\
        -2 & 3
    \end{matrix}
\right |
+ 1
\left |
    \begin{matrix}
        2 & 0\\
        -2 & 1
    \end{matrix}
\right |
=
4*(-2) - 10 + 2
=
-16
$

$|detA_{2}(b)| = 2
\left |
    \begin{matrix}
        2 & 2\\
        -2 & 3
    \end{matrix}
\right |
-4
\left |
    \begin{matrix}
        -1 & 2\\
        3 & 3
    \end{matrix}
\right |
+ 1
\left |
    \begin{matrix}
        -1 & 2\\
        3 & -2
    \end{matrix}
\right |
=
2*10 - 4(-9) + (-4)
=
20 + 36 - 4
=
52
$

$|detA_{3}(b)| = 2
\left |
    \begin{matrix}
        0 & 2\\
        1 & -2
    \end{matrix}
\right |
- 1
\left |
    \begin{matrix}
        -1 & 2\\
        3 & -2
    \end{matrix}
\right |
+ 4
\left |
    \begin{matrix}
        -1 & 0\\
        3 & 1
    \end{matrix}
\right |
=
2*(-2) - (-4) + 4*(-1)
=
-4 + 4 - 4
= -4
$

Since $|detA|$ is non-zero the system has a unique solution, by applying Cramer's rule we get solutions:
\begin{flalign*}
    x_{1} = \frac{detA_{1}(b)}{detA} = \frac{-16}{4} = -4\\
    x_{2} = \frac{detA_{2}(b)}{detA} = \frac{52}{4} = 13\\
    x_{3} = \frac{detA_{3}(b)}{detA} = \frac{-4}{4} = -1
\end{flalign*}

\subsection*{c}
$
\vec{a} = (-3,4), \hspace{5 mm} \vec{b} = (2,5), \hspace{5 mm}
A = 
\left (
    \begin{matrix}
        1 & 2\\
        -2 & 6
    \end{matrix}
\right )
$

We apply theorem about linear transformations and paralellogram (theorem 10, p.223 in our book).\\
$Area(T(S)) = |detA| \dot Area(S)$

$|detA| = 6-(-4) = 10$\\
Area of parallellogram is computed by absolute value of the determinant from the matrix $C$ formed by $\vec{a}$ and $\vec{b}$.
$C =
\left (
    \begin{matrix}
        -3 & 4\\
        2 & 5
    \end{matrix}
\right )
$

$|detC| = -15 - 8 = |-23| = 23$

$10*23 = 230$

Area of S is 230

\newpage

\section*{Exercise 4}

\subsection*{a}
$
\left (
    \begin{matrix}
        4\\
        1\\
        -3\\
        2
    \end{matrix}
\right )
, \hspace{5 mm}
\left (
    \begin{matrix}
        -2\\
        3\\
        4\\
        0
    \end{matrix}
\right )
, \hspace{5 mm}
\left (
    \begin{matrix}
        3\\
        -1\\
        2\\
        0
    \end{matrix}
\right )
, \hspace{5 mm}
\left (
    \begin{matrix}
        1\\
        2\\
        -5\\
        2
    \end{matrix}
\right )
$
        
Finding a linearly independent set is the same as finding the basis.\\
Put all the vectors in the set in a matrix $A$, then row-reduce and see which columns are pivot columns.

$A =
\left (
    \begin{matrix}
        4 & -2 & 3 & 1\\
        1 & 3 & -1 & 2\\
        -3 & 4 & 2 & -5\\
        2 & 0 & 0 & 2
    \end{matrix}
\right )
$

Divide row 1 by 4\\
$
\left (
    \begin{matrix}
        1 & -\frac{1}{2} & \frac{3}{4} & \frac{1}{4}\\
        1 & 3 & -1 & 2\\
        -3 & 4 & 2 & -5\\
        2 & 0 & 0 & 2
    \end{matrix}
\right )
$

Add -1*row1 to row2\\
$
\left (
    \begin{matrix}
        1 & -\frac{1}{2} & \frac{3}{4} & \frac{1}{4}\\
        0 & \frac{7}{2} & -\frac{7}{4} & \frac{7}{4}\\
        -3 & 4 & 2 & -5\\
        2 & 0 & 0 & 2
    \end{matrix}
\right )
$

Add 3*row1 to row3\\
$
\left (
    \begin{matrix}
        1 & -\frac{1}{2} & \frac{3}{4} & \frac{1}{4}\\
        0 & \frac{7}{2} & -\frac{7}{4} & \frac{7}{4}\\
        0 & \frac{5}{2} & \frac{17}{4} & -\frac{17}{4}\\
        2 & 0 & 0 & 2
    \end{matrix}
\right )
$

Add -2*row1 to row4\\
$
\left (
    \begin{matrix}
        1 & -\frac{1}{2} & \frac{3}{4} & \frac{1}{4}\\
        0 & \frac{7}{2} & -\frac{7}{4} & \frac{7}{4}\\
        0 & \frac{5}{2} & \frac{17}{4} & -\frac{17}{4}\\
        0 & 1 & -\frac{3}{2} & \frac{3}{2}
    \end{matrix}
\right )
$

Divide row2 by $\frac{7}{2}$\\
$
\left (
    \begin{matrix}
        1 & -\frac{1}{2} & \frac{3}{4} & \frac{1}{4}\\
        0 & 1 & -\frac{1}{2} & \frac{1}{2}\\
        0 & \frac{5}{2} & \frac{17}{4} & -\frac{17}{4}\\
        0 & 1 & -\frac{3}{2} & \frac{3}{2}
    \end{matrix}
\right )
$

Add $-\frac{5}{2}$*row2 to row3\\
$
\left (
    \begin{matrix}
        1 & -\frac{1}{2} & \frac{3}{4} & \frac{1}{4}\\
        0 & 1 & -\frac{1}{2} & \frac{1}{2}\\
        0 & 0 & \frac{11}{2} & -\frac{11}{2}\\
        0 & 1 & -\frac{3}{2} & \frac{3}{2}
    \end{matrix}
\right )
$

Add -1*row2 to row4\\
$
\left (
    \begin{matrix}
        1 & -\frac{1}{2} & \frac{3}{4} & \frac{1}{4}\\
        0 & 1 & -\frac{1}{2} & \frac{1}{2}\\
        0 & 0 & \frac{11}{2} & -\frac{11}{2}\\
        0 & 0 & -1 & 1
    \end{matrix}
\right )
$

Add $\frac{1}{2}$*row2 to row1\\
$
\left (
    \begin{matrix}
        1 & 0 & \frac{1}{2} & \frac{1}{2}\\
        0 & 1 & -\frac{1}{2} & \frac{1}{2}\\
        0 & 0 & \frac{11}{2} & -\frac{11}{2}\\
        0 & 0 & -1 & 1
    \end{matrix}
\right )
$

Divide row3 by $\frac{11}{2}$\\
$
\left (
    \begin{matrix}
        1 & 0 & \frac{1}{2} & \frac{1}{2}\\
        0 & 1 & -\frac{1}{2} & \frac{1}{2}\\
        0 & 0 & 1 & -1\\
        0 & 0 & -1 & 1
    \end{matrix}
\right )
$

Add 1*row3 to row4\\
$
\left (
    \begin{matrix}
        1 & 0 & \frac{1}{2} & \frac{1}{2}\\
        0 & 1 & -\frac{1}{2} & \frac{1}{2}\\
        0 & 0 & 1 & -1\\
        0 & 0 & 0 & 0
    \end{matrix}
\right )
$

Add $\frac{1}{2}$*row3 to row2\\
$
\left (
    \begin{matrix}
        1 & 0 & \frac{1}{2} & \frac{1}{2}\\
        0 & 1 & 0 & 0\\
        0 & 0 & 1 & -1\\
        0 & 0 & 0 & 0
    \end{matrix}
\right )
$

Add $-\frac{1}{2}$*row3 to row1\\
$
\left (
    \begin{matrix}
        1 & 0 & 0 & 1\\
        0 & 1 & 0 & 0\\
        0 & 0 & 1 & -1\\
        0 & 0 & 0 & 0
    \end{matrix}
\right )
$

Pivot columns at 1, 2 and 3, so a linearly independent subset of the set is:
$
\left (
    \begin{matrix}
        4\\
        1\\
        -3\\
        2
    \end{matrix}
\right )
, \hspace{5 mm}
\left (
    \begin{matrix}
        -2\\
        3\\
        4\\
        0
    \end{matrix}
\right )
, \hspace{5 mm}
\left (
    \begin{matrix}
        3\\
        -1\\
        2\\
        0
    \end{matrix}
\right )
$

\subsection*{b}

\subsubsection*{1}

$H_{1} \subseteq \mathbb{R}^n$\\
$H_{2} \subseteq \mathbb{R}^n$\\
$H = H_{1} \cap H_{2}$

To prove $H$ is a subspace of $\mathbb{R}^n$ it needs to satisfy the three properties for definition of subspace (p. 236 in our book):\\
\begin{enumerate}[(a)]
    \item \textbf{The zero vector of $\mathbb{R}^n$ is in $H$}\\
			 Since the zero vector of $\mathbb{R}^{n}$ is in $H_{1}$ and $H_{2}$ it will be in $H$ per definition of intersection
    \item \textbf{$H$ is closed under vector addition}\\
			Take two vectors $\vec{u}$ and $\vec{v}$ from $H$, $\vec{u},\vec{v} \subset H_{1}$ and $\vec{u},\vec{v} \subset H_{2}$ . Since we know $H_{1}$ and $H_{2}$ are subspaces of $\mathbb{R}^{n}$ $\vec{u}+\vec{v}$ is closed under vector addition in $H_{1}$ and $H_{2}$ and therefore closed under vector addition in their intersection.
    \item\textbf{ $H$ is closed under multiplication by scalars}\\
			Take any vector $\vec{u} \subset H$ and scalar $c$, all vectors $c\vec{u}$ are in both $H_{1}$ and $H_{2}$ which are closed by scalar multiplication, therefore their intersection is too.
\end{enumerate}

\subsection*{c}

$\alpha = \{\alpha_{1}, \alpha_{2}, \alpha_{3}\}
, \hspace{5 mm}
\beta = \{\beta_{1}, \beta_{2}, \beta_{3}\}$

\newpage
\section*{Exercise 5}

\subsection*{a}

$A=
\left (
    \begin{matrix}
        1 & -1 & 2\\
        2 & -2 & 4\\
        0 & 1 & 1
    \end{matrix}
\right )
$

First we find the eigenvalues, we need to find $|detA|$\\
$|detA|=
\left |
    \begin{matrix}
        (1-\lambda) & -1 & 2\\
        2 & (-2-\lambda) & 4\\
        0 & 1 & (1-\lambda)
    \end{matrix}
\right |
= (1-\lambda)
\left |
    \begin{matrix}
        (-2-\lambda) & 4\\
        1 & (1-\lambda)
    \end{matrix}
\right |
- (-1)
\left |
    \begin{matrix}
        2 & 4\\
        0 & (1-\lambda)
    \end{matrix}
\right |
+ 2
\left |
    \begin{matrix}
        2 & -2-\lambda\\
        0 & 1
    \end{matrix}
\right |\\
=
(1-\lambda)
    (\lambda^{2}+\lambda-6)
+ (2-2\lambda)
+ 4\\
=
\lambda^{2}+\lambda-6-\lambda^{3}-\lambda^{2}+6\lambda + 2 - 2\lambda + 4\\
=
-\lambda^{3}+5\lambda\\
$

The eigenvalues are:
$\lambda_{1} = 0
, \hspace{5 mm}
\lambda_{2} = \sqrt{5}
, \hspace{5 mm}
\lambda_{3} = -\sqrt{5}
$

To find the eigenvectors we have to solve the equation $(A-\lambda_{i}I)\vec{v} = 0$ for $i = 1,2,3$

We start with $i=1, \lambda_{1} = 0$\\
$
\left (
    \begin{matrix}
        (1-0) & -1 & 2 \\
        2 & (-2-0) & 4 \\
        0 & 1 & (1-0)
    \end{matrix}
\right )
\vec{v}_{1} = \vec{0}
\rightarrow \\
\left (
    \begin{matrix}
        1 & -1 & 2\\
        2 & -2 & 4\\
        0 & 1 & 1
    \end{matrix}
\right )
\left (
    \begin{matrix}
        x\\
        y\\
        z
    \end{matrix}
\right )
=
\left (
    \begin{matrix}
        0\\
        0\\
        0
    \end{matrix}
\right )
\overrightarrow{row-ops}
\left (
	\begin{matrix}
		1 & 0 & 3\\
		0 & 1 & 1\\
		0 & 0 & 0
	\end{matrix}
\right )
\left (
    \begin{matrix}
        x\\
        y\\
        z
    \end{matrix}
\right )
=
\left (
    \begin{matrix}
        0\\
        0\\
        0
    \end{matrix}
\right )
$

We get the system:\\
$
x = -3z\\
y = -z\\
z free
$

Choose $z=1$ and we get:\\
$
x = -3, \hspace{5 mm} y = -1, \hspace{5 mm}z = 1
$

Which gives us the eigenvector
$\vec{v}_{1} = 
\left (
    \begin{matrix}
        -3\\
        -1\\
        1
    \end{matrix}
\right )
$

Now for $i=2, \lambda_{2}=\sqrt{5}$\\
$
\left (
    \begin{matrix}
        (1-\sqrt{5}) & -1 & 2\\
        2 & (-2-\sqrt{5}) & 4\\
        0 & 1 & (1-\sqrt{5})
    \end{matrix}
\right )
\vec{v}_{2} = \vec{0} \rightarrow\\
\left (
    \begin{matrix}
        (1-\sqrt{5}) & -1 & 2\\
        2 & (-2-\sqrt{5}) & 4\\
        0 & 1 & (1-\sqrt{5})
    \end{matrix}
\right )
\left (
    \begin{matrix}
        x\\
        y\\
        z
    \end{matrix}
\right )
=
\left (
    \begin{matrix}
        0\\
        0\\
        0
    \end{matrix}
\right )
$

Now for $i=3, \lambda_{3}=-\sqrt{5}$\\
$
\left (
    \begin{matrix}
        (1+\sqrt{5}) & -1 & 2\\
        2 & (-2+\sqrt{5}) & 4\\
        0 & 1 & (1+\sqrt{5})
    \end{matrix}
\right )
\vec{v}_{3} = \vec{0} \rightarrow\\
\left (
    \begin{matrix}
        (1+\sqrt{5}) & -1 & 2\\
        2 & (-2+\sqrt{5}) & 4\\
        0 & 1 & (1+\sqrt{5})
    \end{matrix}
\right )
\left (
    \begin{matrix}
        x\\
        y\\
        z
    \end{matrix}
\right )
=
\left (
    \begin{matrix}
        0\\
        0\\
        0
    \end{matrix}
\right )
$

\subsection*{b}

$B =
\left (
	\begin{matrix}
		0 & -4 & -6\\
		-1 & 0 & -3\\
		1 & 2 & 5
	\end{matrix}
\right )
$

We need to find an invertible matrix $P$ and a diagonal matrix $D$ such that $B = PDP^{-1}$

Eigenvalues of B:


\newpage
\section*{Exercise 6}

\subsection*{a}
$
(AB)^{T}\\
((AB)^{T})^{-1} = \\
((AB)^{-1})^{T} = \\
(B^{-1}A^{-1})^{T} = \\
(A^{-1})^{T}(B^{-1})^{T}
$

Alt:\\
$
(AB)^{T} = B^{T}A^{T}\\
(B^{T}A^{T})^{-1} = \\
(A^{T})^{-1}(B^{T})^{-1} = \\
(A^{-1})^{T}(B^{-1})^{T}
$

\subsection*{c}

$T$ is a triangle with vertices\\
$P_{1} = (x_{1}, y_{1}), \hspace{5 mm}
P_{2} = (x_{2}, y_{2}), \hspace{5 mm}
P_{3} = (x_{3}, y_{3}
$

$|detM| =
\left (
	\begin{matrix}
		x_{1} & y_{1} & 1\\
		x_{2} & y_{2} & 1\\
		x_{3} & y_{3} & 1
	\end{matrix}
\right )
=
$
\end{document}

