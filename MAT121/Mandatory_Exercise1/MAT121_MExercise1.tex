\documentclass{article}

\usepackage{parskip}
\usepackage{mathtools}

\begin{document}

\section*{Exercise 1}

\subsection*{a}
$A=
\left (
    \begin{matrix}
        1 & 2 & 0 & 2\\
        0 & 1 & 1 & 1\\
        0 & -2 & 1 & 1\\
        1 & 2 & 1 & 3
    \end{matrix}
\right )
$
 $\vec{b}=
\left (
    \begin{matrix}
        5\\
        3\\
        0\\
        7
    \end{matrix}
\right )
$

$B=
\left [
        \begin{matrix}
            1 & 2 & 0 & 2 & 5\\
            0 & 1 & 1 & 1 & 3\\
            0 & -2 & 1 & 1 & 0\\
            1 & 2 & 1 & 3 & 7
        \end{matrix}
\right ]
$

Add $-1*row1$ to $row4$\\
$
\left [
        \begin{matrix}
            1 & 2 & 0 & 2 & 5\\
            0 & 1 & 1 & 1 & 3\\
            0 & -2 & 1 & 1 & 0\\
            0 & 0 & 1 & 1 & 2
        \end{matrix}
\right ]
$

Add $-2*row2$ to $row1$\\
$
\left [
        \begin{matrix}
            1 & 0 & -2 & 0 & -1\\
            0 & 1 & 1 & 1 & 3\\
            0 & -2 & 1 & 1 & 0\\
            0 & 0 & 1 & 1 & 2
        \end{matrix}
\right ]
$

Add $2*row2$ to $row3$\\
$
\left [
        \begin{matrix}
            1 & 0 & -2 & 0 & -1\\
            0 & 1 & 1 & 1 & 3\\
            0 & 0 & 3 & 3 & 6\\
            0 & 0 & 1 & 1 & 2
        \end{matrix}
\right ]
$

Divide $row3$ by 3\\
$
\left [
        \begin{matrix}
            1 & 0 & -2 & 0 & -1\\
            0 & 1 & 1 & 1 & 3\\
            0 & 0 & 1 & 1 & 2\\
            0 & 0 & 1 & 1 & 2
        \end{matrix}
\right ]
$

Add $2*row3$ to $row1$\\
$
\left [
        \begin{matrix}
            1 & 0 & 0 & 2 & 3\\
            0 & 1 & 1 & 1 & 3\\
            0 & 0 & 1 & 1 & 2\\
            0 & 0 & 1 & 1 & 2
        \end{matrix}
\right ]
$

Add $-1*row3$ to $row2$\\
$
\left [
        \begin{matrix}
            1 & 0 & 0 & 2 & 3\\
            0 & 1 & 0 & 0 & 1\\
            0 & 0 & 1 & 1 & 2\\
            0 & 0 & 1 & 1 & 2
        \end{matrix}
\right ]
$

Add $-1*row3$ to $row4$\\
$
\left [
        \begin{matrix}
            1 & 0 & 0 & 2 & 3\\
            0 & 1 & 0 & 0 & 1\\
            0 & 0 & 1 & 1 & 2\\
            0 & 0 & 0 & 0 & 0
        \end{matrix}
\right ]
$

Pivot columns are 1,2 and 3.\\
Pivot positions are $B_{1,1}$, $B_{2,2}$, $B_{3,3}$. 

Solving $A\Vec{x}=\Vec{b}$. We look at the reduced echelon form of $B$ and that gives us this system:
\begin{flalign*}
x_{1} & & + 2x_{4} &= 3 &\\
    &x_{2} & &= 1\\
      & &x_{3} + x_{4} &= 2 &\\
\end{flalign*}

Gives us solution:
\begin{flalign*}
x_{1} &= 3-2x_{4} &\\
x_{2} &= 1 &\\
x_{3} &= 2-x_{4} &\\
x_{4} &= free
\end{flalign*}

\subsection*{b}

\subsubsection*{Basis and dimension for Col(B)}
Basis: $\{ \{1,0,0,0\}, \{0,1,0,0\}, \{0,0,1,0\} \}$\\
Dimension: by Theorem dim(Col(B)) =\# of vectors in basis = 3

\subsubsection*{Basis and dimension for Row(B)}
Basis: $\{ \{1,0,0,2,3\}, \{0,1,0,0,1\}, \{0,0,1,1,2\}\}\\$
Dimension: by Theorem dim(Row(B)) = dim(Col(B)) = 3

\subsubsection*{Basis and dimension for Null(B)}
Basis: $\{ \{-2, 0, -1, 1\}, \{-3,-1,-2,0,1\} \}$\\
Solve $Bx=0$, we already have reduced echelon form of $A$ by $B$ so just add column in $B$ with all 0s\\

$
\left [
        \begin{matrix}
            1 & 0 & 0 & 2 & 3 & 0\\
            0 & 1 & 0 & 0 & 1 & 0\\
            0 & 0 & 1 & 1 & 2 & 0\\
            0 & 0 & 0 & 0 & 0 & 0
        \end{matrix}
\right ]
$

$
\left [
    \begin{matrix}
        -2x_{4}-3x_{5}\\
         -x_{5}\\
         -x_{4}-2x_{5}\\
         x_{4}\\
         x_{5}
    \end{matrix}
\right ]
$

Decomposition:
$
\left [
    \begin{matrix}
        -2x_{4}-3x_{5}\\
         -x_{5}\\
         -x_{4}-2x_{5}\\
         x_{4}\\
         x_{5}
    \end{matrix}
\right ]
$
$ = x_{4} 
\left [
    \begin{matrix}
        -2\\
        0\\
         -1\\
         1\\
         0
    \end{matrix}
\right ]
+ x_{5}
\left [
    \begin{matrix}
        -3\\
        -1\\
        -2\\
        0\\
        1
    \end{matrix}
\right ]
$

Dimension: 2

\subsubsection*{Rank of B}
By Theorem
\begin{flalign*}
    rank(B) + dim(Null(B)) &= n &\\
    rank(B) &= n-dim(Null(B))\\
    rank(B) &= 5-2\\
    rank(B) &= 3
\end{flalign*}

\subsection*{c}

A matrix $A$ is invertible if its row-reduced echelon form is $I_{4}$, we already know the row-reduced echelon form  of $B$ so row-reduced echelon form of $A$ is just $B$ without the last column. This is because while row-reducing $B$ you also row-reduce $A$.

RREF of $A$:\\
$A=
\left [
    \begin{matrix}
        1 & 0 & 0 & 2\\
        0 & 1 & 0 & 0\\
        0 & 0 & 1 & 1\\
        0 & 0 & 0 & 0
    \end{matrix}
\right ]
$

We see that $A \not = I_{4}$ so $A$ is not invertible.

\newpage

\section*{Exercise 2}

\subsection*{a}

$A=
\left (
    \begin{matrix}
        1 & -1 & 2\\
        0 & 1 & 3
    \end{matrix}
\right )
,
B=
\left (
    \begin{matrix}
        2 & -1\\
        -1 & 0\\
        3 & -1
    \end{matrix}
\right )
, \vec{x} =
\left (
    \begin{matrix}
        1\\
        2
    \end{matrix}
\right )
$

\subsubsection*{y=B$\vec{x}$}
$B\vec{x} =
\left (
    \begin{matrix}
        2 & -1\\
        -1 & 0\\
        3 & -1
    \end{matrix}
\right )
\left (
    \begin{matrix}
        1\\
        2
    \end{matrix}
\right )
=
\left (
    \begin{matrix}
        2 + (-2)\\
        -1 + 0\\
        3 - 2
    \end{matrix}
\right )
=
\left (
    \begin{matrix}
        0\\
        -1\\
        1
    \end{matrix}
\right )
$\\
$\vec{y} =
\left (
    \begin{matrix}
        0\\
        -1\\
        1
    \end{matrix}
\right )
$

\subsubsection*{C=AB}
$AB=
\left (
    \begin{matrix}
        1 & -1 & 2\\
        0 & 1 & 3
    \end{matrix}
\right )
\left (
    \begin{matrix}
        2 & -1\\
        -1 & 0\\
        3 & -1
    \end{matrix}
\right )
=
\left (
    \begin{matrix}
        1*2 + (-1)*(-1) + 2*3 & 1*(-1) + (-1)*0 + 2*(-1)\\
        0*2 + 1*(-1) + 3*3 & 0*(-1) + 1*0 + 3*(-1)
    \end{matrix}
\right )
=
\left (
    \begin{matrix}
        2 + 1 + 6 & -1 + 0 - 2\\
        0 - 1 + 9 & 0 + 0 - 3
    \end{matrix}
\right )
=
\left (
    \begin{matrix}
        9 & -3\\
        8 & -3
    \end{matrix}
\right )
$

$C=
\left (
    \begin{matrix}
        9 & -3\\
        8 & -3
    \end{matrix}
\right )
$

\subsubsection*{A$\vec{y}$}
$A\vec{y} =
\left (
    \begin{matrix}
        1 & -1 & 2\\
        0 & 1 & 3
    \end{matrix}
\right )
\left (
    \begin{matrix}
        0\\
        -1\\
        1
    \end{matrix}
\right )
=
\left (
    \begin{matrix}
        0 + (-1)*(-1) + 2*1\\
        0 + (-1)*1 + 3*1
    \end{matrix}
\right )
=
\left (
    \begin{matrix}
        0 + 1 + 2\\
        0 - 1 + 3
    \end{matrix}
\right )
=
\left (
    \begin{matrix}
        3\\
        2
    \end{matrix}
\right )
$

$A\vec{y}=
\left (
    \begin{matrix}
        3\\
        2
    \end{matrix}
\right )
$

\subsubsection*{C$\vec{x}$}
$C\vec{x} =
\left (
    \begin{matrix}
        9 & -3\\
        8 & -3
    \end{matrix}
\right )
\left (
    \begin{matrix}
        1\\
        2
    \end{matrix}
\right )
=
\left (
    \begin{matrix}
        9 + (-3)*2\\
        8 + (-3)*2
    \end{matrix}
\right )
=
\left (
    \begin{matrix}
        9 - 6\\
        8 - 6
    \end{matrix}
\right )
=
\left (
    \begin{matrix}
        3\\
        2
    \end{matrix}
\right )
$

$C\vec{x}=
\left (
    \begin{matrix}
        3\\
        2
    \end{matrix}
\right )
$

\subsubsection*{C$^{T}$B$^{T}$}
C$^{T}=
\left (
    \begin{matrix}
        9 & 8\\
        -3 & -3
    \end{matrix}
\right )
$,
B$^{T}=
\left (
    \begin{matrix}
        2 & -1 & 3\\
        -1 & 0 & -1\\
    \end{matrix}
\right )
$

C$^{T}$B$^{T}=
\left (
    \begin{matrix}
       9*2 + 8*(-1) & 9*(-1) + 0 & 9*3 + 8*(-1)\\
       -3*2 + (-3)*(-1) & (-3)*(-1) + 0 & (-3)*3 + (-3)*(-1)
    \end{matrix}
\right )
=
\left (
    \begin{matrix}
        18-8 & -9 & 27-8\\
        -6+3 & 3 & -9+3
    \end{matrix}
\right )
=
\left (
    \begin{matrix}
        10 & -9 & 19\\
        -3 & 3 & -6
    \end{matrix}
\right )
$

C$^{T}$B$^{T}=
\left (
    \begin{matrix}
        10 & -9 & 19\\
        -3 & 3 & -6
    \end{matrix}
\right )
$

\subsubsection*{(A$^{T}$+B)C}
A$^{T}=
\left (
    \begin{matrix}
        1 & 0\\
        -1 & 1\\
        2 & 3
    \end{matrix}
\right )
$

A$^{T}C=
\left (
    \begin{matrix}
        1 & 0\\
        -1 & 1\\
        2 & 3
    \end{matrix}
\right )
\left (
    \begin{matrix}
        9 & -3\\
        8 & -3
    \end{matrix}
\right )
=
\left (
    \begin{matrix}
        9 + 0 & -3 + 0\\
        -9 + 8 & 3 - 3\\
        18 + 24 & -6 - 9
    \end{matrix}
\right )
=
\left (
    \begin{matrix}
        9 & -3\\
        -1 & 0\\
        42 & -15
    \end{matrix}
\right )
$

We notice that $(C^{T}B^{T})^{T} = BC$ by transpose properties $(A^{T})^{T}=A$ and $(AB)^{T}=B^{T}A^{T}$.\\
$(C^{T}B^{T})^{T} = BC =
\left (
    \begin{matrix}
        10 & -9 & 19\\
        -3 & 3 & -6
    \end{matrix}
\right )
^{T}=
\left (
    \begin{matrix}
        10 & -3\\
        -9 & 3\\
        19 & -6
    \end{matrix}
\right )
$

A$^{T}$C + BC =
$
\left (
    \begin{matrix}
        9 & -3\\
        -1 & 0\\
        42 & -15
    \end{matrix}
\right )
+
\left (
    \begin{matrix}
        10 & -3\\
        -9 & 3\\
        19 & -6
    \end{matrix}
\right )       
=
\left (
    \begin{matrix}
        19 & -6\\
        -10 & 3\\
        61 & -21
    \end{matrix}
\right )
$

\subsection*{b}

\subsubsection*{$A^{2}$?}
We cannot calculate $A^{2}$ because to multiply two matrices $A$ with dimensions $m \times k$ the second matrix $B$ needs to have dimensions $k \times n$ by theorem. In this exercise $B=A$ and since $A$ is not a square matrix it will not meet the requirements.

\subsubsection*{$A^{-1}$?}
We cannot calculate $A^{-1}$ because $A$ is not square, therefore $detA$ is not defined and by theorem it does not have an inverse

\subsection*{c}
We denote the matrix as $A$.\\
$A=
\left (
    \begin{matrix}
        1 & 1 & 2\\
        -1 & 1 & 1\\
        2 & 3 & 3
    \end{matrix}
\right )
$

We apply formula for inverse of $n \times n$ matrix:\\
$A^{-1} = \frac{1}{det(A)} adj(A)$

To find $adj(A)$ we need to find transpose of the cofactor-matrix:

Cofactor-matrix $C$:\\
$A_{11}=
\left |
    \begin{matrix}
        1 & 1\\
        3 & 3
    \end{matrix}
\right |
, A_{12} =
\left |
    \begin{matrix}
        -1 & 1\\
        2 & 3
    \end{matrix}
\right |
, A_{13} =
\left |
    \begin{matrix}
        -1 & 1\\
        2 & 3
    \end{matrix}
\right |
$

$A_{21}=
\left |
    \begin{matrix}
        1 & 2\\
        3 & 3
    \end{matrix}
\right |
, A_{22}=
\left |
    \begin{matrix}
        1 & 2\\
        2 & 3
    \end{matrix}
\right |
, A_{23}=
\left |
    \begin{matrix}
        1 & 1\\
        2 & 3
    \end{matrix}
\right |
$

$A_{31}=
\left |
    \begin{matrix}
        1 & 2\\
        1 & 1
    \end{matrix}
\right |
, A_{32}=
\left |
    \begin{matrix}
        1 & 2\\
        -1 & 1
    \end{matrix}
\right |
, A_{33}=
\left |
    \begin{matrix}
        1 & 1\\
        -1 & 1
    \end{matrix}
\right |
$

$
\left (
    \begin{matrix}
        + & - & +\\
        - & + & -\\
        + & - & +
    \end{matrix}
\right )
$

$C=
\left (
    \begin{matrix}
        \left |
            \begin{matrix}
                1 & 1\\
                3 & 3
            \end{matrix}
        \right |
        = + 0,
        \left |
            \begin{matrix}
                -1 & 1\\
                2 & 3
            \end{matrix}
        \right |
        = - (-5),
        \left |
            \begin{matrix}
                -1 & 1\\
                2 & 3
            \end{matrix}
        \right |
        = + (-5)\\
        \left |
            \begin{matrix}
                1 & 2\\
                3 & 3
            \end{matrix}
        \right |
        = - (-3),
        \left |
            \begin{matrix}
                1 & 2\\
                2 & 3
            \end{matrix}
        \right |
        = + (-1),
        \left |
            \begin{matrix}
                1 & 1\\
                2 & 3
            \end{matrix}
        \right |
        = - 1\\
        \left |
            \begin{matrix}
                1 & 2\\
                1 & 1
            \end{matrix}
        \right |
        = + (-1),
        \left |
            \begin{matrix}
                1 & 2\\
                -1 & 1
            \end{matrix}
        \right |
        = - 3,
        \left |
            \begin{matrix}
                1 & 1\\
                -1 & 1
            \end{matrix}
        \right |
        = + 2
    \end{matrix}
\right )
=\\
\left (
    \begin{matrix}
        0 & 5 & -5\\
        3 & -1 & -1\\
        -1 & -3 & 2
    \end{matrix}
\right )
$

We calculate det$A$ next by multiplying each element at index $i$ in row $A_{1}$ with the element in index $i$ in row $C_{1}$ and summing them together:\\
det$A = (1*0) + (1*5) + (2*-5) = 0 + 5 - 10 = -5$

adj(A) = C$^{T}$
C$^{T}=
\left (
    \begin{matrix}
        0 & 3 & -1\\
        5 & -1 & -3\\
        -5 & -1 & 2
    \end{matrix}
\right )
$

A$^{-1} = \frac{1}{-5}
\left (
    \begin{matrix}
        0 & 3 & -1\\
        5 & -1 & -3\\
        -5 & -1 & 2
    \end{matrix}
\right )       
=
\left (
    \begin{matrix}
        0 & \frac{3}{-5} & \frac{1}{5}\\
        -1 & \frac{1}{5} & \frac{3}{5}\\
        1 & \frac{1}{5} & \frac{2}{-5}
    \end{matrix}
\right )
$

A$^{-1} =
\left (
    \begin{matrix}
        0 & \frac{3}{-5} & \frac{1}{5}\\
        -1 & \frac{1}{5} & \frac{3}{5}\\
        1 & \frac{1}{5} & \frac{2}{-5}
    \end{matrix}
\right )
$
\newpage

\section*{Exercise 3}

\subsection*{a}

\end{document}
